\documentclass[9pt, technote]{IEEEtran}
\IEEEoverridecommandlockouts

\usepackage{cite}
\usepackage{amsmath,amssymb,amsfonts}
\usepackage{algorithmic}
\usepackage{graphicx}
\usepackage{textcomp}
\usepackage{xcolor}
\usepackage{amsmath}
\usepackage{url}
\def\BibTeX{{\rm B\kern-.05em{\sc i\kern-.025em b}\kern-.08em
    T\kern-.1667em\lower.7ex\hbox{E}\kern-.125emX}}
\begin{document}

\title{Code Management and Generative AI Best Practices\\
    \large A Collaborative Review\\
    \large EE450 Military Robotic Applications

}

% When you contribute to this document, add an \author section for yourself
\author{\IEEEauthorblockN{Cadet 1}\\
\IEEEauthorblockA{\textit{United States Military Academy}\\
West Point, NY \\
First.Last@westpoint.edu}\\
\and
\IEEEauthorblockN{Cadet 2}\\
\IEEEauthorblockA{\textit{United States Military Academy}\\
West Point, NY \\
First.Last@westpoint.edu}
}

\maketitle

\begin{abstract}
This is a collection of best-practice proposals by the cadets of EE450 Military Robotic Applications.
The cadets use this document to share their perspectives, experiences, and recommendations on the use
of software IDEs (specifically, Visual Studio Code and the Arduino IDE), robotics project code structure,
and the use of generative artificial intelligence in completing robotics projects. This is a
"by-cadets-for-cadets" living record that both encourages reflection and guides future cadets in their
robotics endeavors.
\end{abstract}
\section{Instructions to Contributors}
This document is divided into four parts:
\begin{enumerate}
    \item Getting Started with Arduino Programming
    \item Code Management and Structure
    \item Integrating Generative AI
    \item Testimonials and Examples
\end{enumerate}

You may contribute to any or all of these sections as much or as little as you like. It is critical that
your contributions remain \textit{high quality} and \textit{easy to understand}. Do your best to place
your inputs in the correct sections and subsections, but feel free to create your own sections or subsections
if you think you need to. You must not, under any circumstances, provide direct answers to any of the course
projects, mini-projects, quizzes, or other assignments --- the purpose is to guide other cadets on their
own problem-solving, not to solve the problem for them.

Contributions to this document must be made via a \textit{pull request} to the appropriate GitHub repository.
This first requires you to \textit{clone} the project repository and create a new \textit{branch}. If you need
guidance on how GitHub works, or how to submit a pull request, you can check the provided references.\cite{github_branch} \cite{github_pull}
\section{Getting Started with Arduino Programming}
% Use this section to discuss how to set up the IDE, how to install libraries, and how to upload code to the Arduino microcontroller.
% Any generic guidance not specific to one IDE can go here, before the subsections.
\subsection{Visual Studio Code}

\subsection{Arduino IDE}

\section{Code Management and Structure}
% Below are a few recommended subsections for you to contribute to, but feel free to create your own if you think it is appropriate.
\subsection{Incorporating the Sense-Decide-Act Paradigm} 
% Did you structure your programs this way? Was it successful or not?

\subsection{Implementing States and Using State Diagrams} 
% Are state diagrams helpful at the start of a project? How did you use them?

\subsection{File Structure and Version Management --- Avoid Drowning in Code} 
% Arduino sketches can grow into large, complex files. How did you organize them internally so you are always confident
% that your program is doing what you want?

\section{Integrating Generative AI}

\subsection{Registering for Copilot Pro} 
% provide instructions for registering for Copilot Pro using a student license

\subsection{Managing and Integrating Generated Code} 
% give guidance on how to prompt Copilot -- how do you structure the prompt, what context do you give it, and what do you
% do with what it generates?

\subsection{Documentation and Citation of Generated Code} 
% How do you document and cite generated code? How do you avoid plagiarism and ensure all generated code is clearly marked?

\section{Testimonials and Examples}
Feel free to add any other guidance, examples (good or bad!), or advice to other cadets here.

\subsection{Advice from CDT X} % example
If I had to give one piece of advice, it's to research the sensors I'm using and understand how they work, \textit{then}
build my state diagram and think about what I actually want the robot to do in terms of sensors and actuators, \textit{then}
think about how I want to prompt the generative AI. Starting by inputting the problem statement into Copilot never worked
for me a single time --- it always gave me something I didn't understand and had no idea how to fix, causing problems later.

\bibliographystyle{plain}
\bibliography{References}


\end{document}
